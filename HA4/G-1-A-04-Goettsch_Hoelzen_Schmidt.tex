\documentclass[12pt, paper=a4]{article}
\usepackage[utf8]{inputenc}
\usepackage[german]{babel}
\usepackage{mathrsfs}
\usepackage{amsmath}
\usepackage{amssymb}
\usepackage{listings}
\usepackage{graphicx}
\usepackage{fancyhdr}

\setlength{\parindent}{0pt}

\author{Mareike Göttsch, 6695217, Gruppe 2\\Paul Hölzen, 6673477, Gruppe 1\\Sven Schmidt, 6217064, Gruppe 1}

\title{FGI 2 Hausaufgaben 4}

\rhead{M. Göttsch, G-2; P. Hölzen, G-1; S. Schmidt, G-1}
\pagestyle{fancy}
\begin{document}
\maketitle

\section*{Aufgabe 4.3}

\subsection*{1.}
$L(TS_{Enterprise}) = ((du) + (l(kp^*bw)^*e))^*$\\
$L^\omega(TS_{Enterprise}) = ((du) + (l(kp^*bw)^*e))^\omega$\\

\subsection*{2.}
$SS(M_{Enterprise}) = ((01) + (02(3^*42)^*))^\omega$\\

\subsection*{3.}
Seien die Etiketten von $M_{Enterprise}$:\\
$E_S(s_0) = \{orbit\}$\\
$E_S(s_1) = \{away, orbit\}$\\
$E_S(s_2) = \{warp\}$\\
$E_S(s_3) = \{shields\}$\\
$E_S(s_4) = \emptyset$\\

Die Etikettensprache ist dann durch Einsetzen in die Menge aller Pfade (4.3.2):\\
\begin{align*}
E_S(SS(M_{Enterprise})) &= E_S(((01) + (02(3^*42)^*))^\omega)\\
&= ((E_S(s_0)E_S(s_1)) + (E_S(s_0)E_S(s_2)(E_S(s_3)^*E_S(s_4)E_S(s_2))^*))^\omega\\
&= ((\{orbit\}\{away, orbit\}) + (\{orbit\}\{warp\}(\{shields\}^*\emptyset\{warp\})^*))^\omega
\end{align*}

\subsection*{4.}
$Sat(shields) = \{s_3\}$\\
$Sat(\neg orbit) = \{s_2, s_3, s_4\}$\\
$Sat(warp) = \{s_2\}$\\

Die Formel bedeutet: ``Folgendes gilt immer: Wenn die Schilde aktiv sind, dann wird, wenn sie im nächsten Schritt deaktiviert werden, irgendwann einmal der Warp eingeschaltet.''\\

Beweis:\\
Die Formel $f$ macht eine Aussage, die zu jedem Zeitpunkt gelten soll. Die Prämisse der in diesen $G$-Operator geschachtelten Implikation ist $shields$. Da diese atomare Aussage nur im Zustand $s_3$ wahr ist, gilt die Implikation in jedem anderen Zustand, der zuvor besucht wird.\\
Sobald, das erste Mal $s_3$ besucht wird, sagt die erste Implikation aus, dass nun die Teilformel $(X \neg shields) \Rightarrow F warp$ gilt. Sprich, wenn im nächsten Schritt $\neg shields$ gilt, dann gilt $F warp$. Bleibt man im Zustand $s_3$ indem man die Kante $p$ benutzt, so gilt die Prämisse nicht und die Formel ist erfüllt. Nimmt man die Kante zu $s_4$ ist die Prämisse erfüllt und da die einzige Möglichkeit weiterzumachen die Kante zu Zustand $s_2$ ist, in dem $warp$ gilt, ist auch die Konklusion erfüllt.\\
Die Formel gilt also im Startzustand $s_0$, da oben beschriebene Zustandsfolge unendlich oft durchlaufen werden kann und somit auch der $G$-Operator erfüllt ist.\\

\subsection*{5.}
Mehr Beweise...\\

\section*{Aufgabe 4.4}

\section*{Aufgabe 4.5}

\end{document}
