\documentclass[12pt, paper=a4]{article}
\usepackage[utf8]{inputenc}
\usepackage[german]{babel}
\usepackage{mathrsfs}
\usepackage{amsmath}
\usepackage{amssymb}
\usepackage{listings}
\usepackage{graphicx}
\usepackage{fancyhdr}

\setlength{\parindent}{0pt}

\author{Mareike Göttsch, 6695217, Gruppe 2\\Paul Hölzen, 6673477, Gruppe 1\\Sven Schmidt, 6217064, Gruppe 1}

\title{FGI 2 Hausaufgaben 4}

\rhead{M. Göttsch, G-2; P. Hölzen, G-1; S. Schmidt, G-1}
\pagestyle{fancy}
\begin{document}
\maketitle

\section*{Aufgabe 4.3}

\subsection*{1.}
$L(TS_{Enterprise}) = ((du) + (l(kp^*bw)^*e))^*$\\
$L^\omega(TS_{Enterprise}) = ((du) + (l(kp^*bw)^*e))^\omega$\\

\subsection*{2.}
$SS(M_{Enterprise}) = ((01) + (02(3^*42)^*))^\omega$\\

\subsection*{3.}
Seien die Etiketten von $M_{Enterprise}$:\\
$E_S(s_0) = \{orbit\}$\\
$E_S(s_1) = \{away, orbit\}$\\
$E_S(s_2) = \{warp\}$\\
$E_S(s_3) = \{shields\}$\\
$E_S(s_4) = \emptyset$\\

Die Etikettensprache ist dann durch Einsetzen in die Menge aller Pfade (4.3.2):\\
\begin{align*}
E_S(SS(M_{Enterprise})) &= E_S(((01) + (02(3^*42)^*))^\omega)\\
&= ((E_S(s_0)E_S(s_1)) + (E_S(s_0)E_S(s_2)(E_S(s_3)^*E_S(s_4)E_S(s_2))^*))^\omega\\
&= ((\{orbit\}\{away, orbit\}) + (\{orbit\}\{warp\}(\{shields\}^*\emptyset\{warp\})^*))^\omega
\end{align*}

\subsection*{4.}
$Sat(shields) = \{s_3\}$\\
$Sat(\neg orbit) = \{s_2, s_3, s_4\}$\\
$Sat(warp) = \{s_2\}$\\

Die Formel bedeutet: ``Folgendes gilt immer: Wenn die Schilde aktiv sind, dann wird, wenn sie im nächsten Schritt deaktiviert werden, irgendwann einmal der Warp eingeschaltet.''\\

Die Formel gilt im Anfangszustand $s_0$. (Hier Beweis einfügen...)\\

\subsection*{5.}
Mehr Beweise...\\

\section*{Aufgabe 4.4}

\section*{Aufgabe 4.5}

\end{document}
