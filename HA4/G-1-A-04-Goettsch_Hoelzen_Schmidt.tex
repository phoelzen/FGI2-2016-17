\documentclass[12pt, paper=a4]{article}
\usepackage[utf8]{inputenc}
\usepackage[german]{babel}
\usepackage{mathrsfs}
\usepackage{amsmath}
\usepackage{amssymb}
\usepackage{listings}
\usepackage{graphicx}
\usepackage{fancyhdr}

\setlength{\parindent}{0pt}

\author{Mareike Göttsch, 6695217, Gruppe 2\\Paul Hölzen, 6673477, Gruppe 1\\Sven Schmidt, 6217064, Gruppe 1}

\title{FGI 2 Hausaufgaben 4}

\rhead{M. Göttsch, G-2; P. Hölzen, G-1; S. Schmidt, G-1}
\pagestyle{fancy}
\begin{document}
\maketitle

\section*{Aufgabe 4.3}

\subsection*{1.}
$L(TS_{Enterprise}) = ((du) + (l(kp^*bw)^*e))^*$\\
$L^\omega(TS_{Enterprise}) = ((du) + (l(kp^*bw)^*e))^\omega$\\

\subsection*{2.}
$SS(M_{Enterprise}) = ((01) + (02(3^*42)^*))^\omega$\\

\subsection*{3.}
Seien die Etiketten von $M_{Enterprise}$:\\
$E_S(s_0) = \{orbit\}$\\
$E_S(s_1) = \{away, orbit\}$\\
$E_S(s_2) = \{warp\}$\\
$E_S(s_3) = \{shields\}$\\
$E_S(s_4) = \emptyset$\\

Die Etikettensprache ist dann durch Einsetzen in die Menge aller Pfade (4.3.2):\\
\begin{align*}
E_S(SS(M_{Enterprise})) &= E_S(((01) + (02(3^*42)^*))^\omega)\\
&= ((E_S(s_0)E_S(s_1)) + (E_S(s_0)E_S(s_2)(E_S(s_3)^*E_S(s_4)E_S(s_2))^*))^\omega\\
&= ((\{orbit\}\{away, orbit\}) + (\{orbit\}\{warp\}(\{shields\}^*\emptyset\{warp\})^*))^\omega
\end{align*}

\subsection*{4.}
$Sat(shields) = \{s_3\}$\\
$Sat(\neg orbit) = \{s_2, s_3, s_4\}$\\
$Sat(warp) = \{s_2\}$\\

Die Formel bedeutet: ``Folgendes gilt immer: Wenn die Schilde aktiv sind, dann wird, wenn sie im nächsten Schritt deaktiviert werden, irgendwann einmal der Warp eingeschaltet.''\\

Beweis:\\
Die Formel $f$ macht eine Aussage, die zu jedem Zeitpunkt gelten soll. Die Prämisse der in diesen $G$-Operator geschachtelten Implikation ist $shields$. Da diese atomare Aussage nur im Zustand $s_3$ wahr ist, gilt die Implikation in jedem anderen Zustand, der zuvor besucht wird.\\
Sobald, das erste Mal $s_3$ besucht wird, sagt die erste Implikation aus, dass nun die Teilformel $(X \neg shields) \Rightarrow F warp$ gilt. Sprich, wenn im nächsten Schritt $\neg shields$ gilt, dann gilt $F warp$. Bleibt man im Zustand $s_3$ indem man die Kante $p$ benutzt, so gilt die Prämisse nicht und die Formel ist erfüllt. Nimmt man die Kante zu $s_4$ ist die Prämisse erfüllt und da die einzige Möglichkeit weiterzumachen die Kante zu Zustand $s_2$ ist, in dem $warp$ gilt, ist auch die Konklusion erfüllt.\\
Die Formel gilt also im Startzustand $s_0$, da oben beschriebene Zustandsfolge unendlich oft durchlaufen werden kann und somit auch der $G$-Operator erfüllt ist.\\

\subsection*{5.}
Mehr Beweise...\\

\section*{Aufgabe 4.4}
Gegeben ist ein unendlicher Zustandspfad $\pi$aus der Menge $010\left(23^{4}4\right)^{\omega}$.

Die zu dem Pfad gehörende Zustandsetikettenfunktion lautet demnach\\
\begin{align*}
ES\left(\pi\right)=&\left\{ \neg a,o,\neg s,\neg w\right\} \left\{ a,o,\neg s,\neg w\right\} \left\{ \neg a,o,\neg s,\neg w\right\} \\
 & \left(\left\{ \neg a,\neg o,\neg s,w\right\} \left\{ \neg a,\neg o,s,\neg w\right\} ^{4}\left\{ \neg a,\neg o,\neg s,\neg w\right\} \right)^{\omega},
  \end{align*}
\\
wobei $a,o,s,w$ Abkürzungen für die Zustandsetiketten $\underline{a}way,\underline{o}rbit,\underline{s}hields,\underline{w}arp$
sind.\\

\begin{tabular}{|c|c|c|}
	\hline 
	$f$ & $M_{Enterprise}\models f$ & $M_{Enterprise},\pi\models f$\tabularnewline
	\hline 
	\hline 
	$\lozenge\square\left(w\lor s\right)$ & 0 & 0\tabularnewline
	\hline 
	$\square\lozenge\left(w\lor s\right)$ & 0 & 1\tabularnewline
	\hline 
	$\square\left(s\Rightarrow\left(sU\left(\Circle w\right)\right)\right)$ & 0 & 0\tabularnewline
	\hline 
	$\square\left(o\Rightarrow\Circle\Circle w\right)$ & 0 & 0\tabularnewline
	\hline 
	$\square\left(\left(\lnot o\land\lnot s\land\lnot w\land\Circle\Circle\lnot o\right)\Rightarrow\lozenge s\right)$ & 1 & 1\tabularnewline
	\hline 
	$\Circle\Circle\Circle\lozenge\left(o\land\lnot w\right)$ & 0 & 0\tabularnewline
	\hline 
\end{tabular}
\\
\\
Grundsaätzlich gilt: $\left(M_{Enterprise}\models f\right)\Rightarrow\left(M_{Enterprise},\pi\models f\right)$,
denn ist eine Funktion für alle Pfade gültig so auch für einen bestimmten.
\begin{enumerate}
	\item Die erste Formel kann natürlichsprachlich durch ``Irgendwann gilt
	immer warp oder shield'' ausgedrückt werden. Da sich auf dem gegebenen
	Pfad $\pi$ die Zustandsetiketten $\left\{ \neg a,\neg o,\neg s,\neg w\right\} $
	unendlich oft wiederholen ist diese Aussage allerdings falsch.
	\item ``Es gilt immer irgendwann warp oder shield''. Diese Aussge wird
	durch die unendlich Wiederholung von $\left\{ \neg a,\neg o,\neg s,w\right\} ,\left\{ \neg a,\neg o,s,\neg w\right\} $
	für den Pfad $\pi$ gültig. Auf einem zweiten Pfad $\varPi=s_{0}s_{1}s_{0}s_{1}...$
	gilt die Aussage allerdings nicht daher istsie auch nicht auf ganz
	$M$ gültig.
	\item Gilt nicht auf $\pi$ da nach $\left\{ \neg a,\neg o,s,\neg w\right\} ^{4}\left\{ \neg a,\neg o,\neg s,\neg w\right\} $
	``shields'' nicht gilt bis warp wahr ist, sondern vorher bereits
	ungültig ist.
	\item Nach $\left\{ \neg a,o,\neg s,\neg w\right\} \left\{ a,o,\neg s,\neg w\right\} \left\{ \neg a,o,\neg s,\neg w\right\} $
	beginnt der Pfad mit der Aussage ``orbit'' ist aber zwei Zustände
	später nicht bei der Aussage ``warp''
	\item Diese Aussage gilt für alle Pfade, da die lange Und-Verknüpfung nur
	dann wahr werden kann, wenn der Pfad die sequenz $\varPi=...s_{4}s_{2}s_{3}...$
	enthält. Dann gian entsprechender Stelle aber auch ``irgendwann shields''.
	\item Nach drei Transitionen gilt niemals wieder die Aussage ``orbit''.
	Die Funktion ist daher nicht gültig auf dem Pfad $\pi.$\end{enumerate}

\section*{Aufgabe 4.5}

\end{document}
