\documentclass[12pt, paper=a4]{article}
\usepackage[utf8]{inputenc}
\usepackage[german]{babel}
\usepackage{mathrsfs}
\usepackage{amsmath}
\usepackage{amssymb}
\usepackage{listings}
\usepackage{graphicx}
\usepackage{fancyhdr}
\usepackage{extarrows}

\setlength{\parindent}{0pt}

\author{Mareike G\"ottsch, 6695217, Gruppe 2\\Paul H\"olzen, 6673477, Gruppe 1\\Sven Schmidt, 6217064, Gruppe 1}

\title{FGI 2 Hausaufgaben 12}

\rhead{M. G\"ottsch, G-2; P. H\"olzen, G-1; S. Schmidt, G-1}
\pagestyle{fancy}
\begin{document}
\maketitle

\section*{Aufgabe 12.3}
\subsection*{1.}
\begin{center}
\begin{tabular}{ c c | c c c c l }
	\multicolumn{2}{l|}{ba(b(a+b)c+ba(b+c))} &  \multicolumn{4}{c}{(b+b)(ab(ac+bc)+ab(ab+ac))} \\
		\multicolumn{2}{c|}{$\downarrow$b} & \multicolumn{4}{c}{ $\downarrow$b $\downarrow$b}\\
	\multicolumn{2}{l|}{a(b(a+b)c+ba(b+c))} & \multicolumn{4}{c}{ab(ac+bc)+ab(ab+ac)}\\
		\multicolumn{2}{c|}{$\downarrow$a} & \multicolumn{2}{c}{ $\downarrow$a} &&\multicolumn{2}{c}{ $\downarrow$a}\\
	\multicolumn{2}{l|}{b(a+b)c+ba(b+c)} & \multicolumn{2}{c}{b(ac+bc)} && \multicolumn{2}{c}{b(ab+ac)} \\
		$\downarrow$b & $\downarrow$b & \multicolumn{2}{c}{ $\downarrow$b} &&\multicolumn{2}{c}{ $\downarrow$b}\\
	(a+b)c & a(b+c) & \multicolumn{2}{c}{ac+bc} && \multicolumn{2}{c}{ab+ac} \\
		$\downarrow$a $\downarrow$b & $\downarrow$a & $\downarrow$a & $\downarrow$b & & $\downarrow$a & $\downarrow$a\\
	c & b+c & c & c & & b & c\\
		$\downarrow$c & $\downarrow$b $\downarrow$c & $\downarrow$c & $\downarrow$c & & $\downarrow$b & $\downarrow$c\\
	$\surd$ & $\surd$ & $\surd$ & $\surd$ & & $\surd$ & $\surd$\\
\end{tabular}
\end{center}


\subsection*{2.}
Es ist c$\overleftrightarrow{\underline{\quad}}$c, (b+c)$\overleftrightarrow{\underline{\quad}}$b und (b+c)$\overleftrightarrow{\underline{\quad}}$c. Damit ist auch (a+b)c $\overleftrightarrow{\underline{\quad}}$ac+bc und a(b+c)$\overleftrightarrow{\underline{\quad}}$ab+ac sowie b(a+b)c+ba(b+c)$\overleftrightarrow{\underline{\quad}}$b(ac+bc) und b(a+b)c+ba(b+c)$\overleftrightarrow{\underline{\quad}}$b(ab+ac). Damit gilt auch a(b(a+b)c+ba(b+c))$\overleftrightarrow{\underline{\quad}}$ab(ac+bc)+ab(ab+ac) und somit schlie{\ss}lich $t_3\overleftrightarrow{\underline{\quad}}t_4$.

\subsection*{3.}
Ja, s.o..
\subsection*{4.}
%fehlt noch

\section*{12.4}

\subsection*{1.}
\rule{1cm}{1pt} \(A_0, \sigma_1\)\\
\(a \xlongrightarrow{\text{a}} \surd\)\\
\rule{2.5cm}{1pt} \(T_{+L}^\surd, \sigma_2\)\\
\((b+a) \xlongrightarrow{\text{a}} \surd\)\\
\rule{5cm}{1pt} \(T_{\bullet}^\surd, \sigma_3\)\\
\((b+a)(cd+d) \xlongrightarrow{\text{a}} (cd+d)\)\\
\rule{5.5cm}{1pt} \(T_{\bullet}, \sigma_4\)\\
\((b+a)(cd+d)b \xlongrightarrow{\text{a}} (cd+d)b\)\\
\rule{8.5cm}{1pt} \(T_{+L}, \sigma_5\)\\
\((da+(a+b)c)+(b+a)(cd+d)b \xlongrightarrow{\text{a}} (cd+d)b\)\\
\rule{11.5cm}{1pt} \(T_{+R}, \sigma_6\)\\
\((da+(a+b)c)+(b+a)(cd+d)b+(a(c+d)+b) \xlongrightarrow{\text{a}} (cd+d)b\)\\

\(\sigma_1: v \mapsto a\)\\
\(\sigma_2: v \mapsto a; y \mapsto a; x \mapsto b\)\\
\(\sigma_3: v \mapsto a; x \mapsto (b+a); y \mapsto (cd+d)\)\\
\(\sigma_4: v \mapsto a; x \mapsto (b+a)(cd+d); x' \mapsto (cd+d); y \mapsto b\)\\
\(\sigma_5: v \mapsto a; y \mapsto (b+a)(cd+d)b; y' \mapsto (cd+d)b; x \mapsto (da+(a+b)c)\)\\
\(\sigma_6: v \mapsto a; x \mapsto (da+(a+b)c)+(b+a)(cd+d)b; x' \mapsto (cd+d)b; \\ y \mapsto (a(c+d)+b)\)\\

\subsection*{2.}
\(t_7 = (\underline{b+b})((ab)(ac+bc)+a((ab+ac)b))\\
\xrightarrow{\text{R3}} b((ab)(ac+bc)+a(\underline{(ab+ac)b)})\\
\xrightarrow{\text{R4}} b(\underline{(ab)(ac+bc)}+a(abb+acb))\\
\xrightarrow{\text{R5}} b(a(\underline{b(ac+bc)})+a(abb+acb))\\
\xrightarrow{\text{R4}} b(\underline{a(bac+bbc)}+\underline{a(abb+acb)})\\
\xrightarrow{\text{R4}} \underline{b(abac+abbc+aabb+aacb)}\\
\xrightarrow{\text{R4}} babac+babbc+baabb+baacb\\
\)

\subsection*{3.}
\(t_8=b(a(a(cb)+a(bb))+a((\underline{b(b+a)})c))
\xrightarrow{\text{R4}} b(a(acb+abb)+a(\underline{(bb+ba)c}))\\
\xrightarrow{\text{R4}} b(\underline{a(acb+abb)}+\underline{a(bbc+bac)})\\
\xrightarrow{\text{R4}} \underline{b(aacb+aabb+abbc+abac)}\\
\xrightarrow{\text{R4}} \underline{baacb+baabb+babbc+babac}\\
\xrightarrow{\text{R1}} babac+babbc+baabb+baacb\\
\)

\(t_7\) und \(t_8\) sind \"aquivalent, da sie auf gleiche Normalformen gebracht werden k\"onnen und sie somit nach Satz 9.17 \"aquivalent sind.

\section*{12.5}




\section*{12.6}
	\begin{itemize}
	\item Der ACP-Kalk\"ul mit gesch\"utzter Rekursion ist korrekt bez\"uglich Bisimulation, aber nicht vollst\"andig.\\
		Wahr oder falsch?\\
		\textit{(Lesestoff Woche 12)}
	\item Die Kommunikationsfunktion $\gamma$ ist weder kommutativ noch assoziativ.\\
		Wahr oder falsch?\\
		\textit{(Lesestoff Woche 12)}
	\end{itemize}
\end{document}
